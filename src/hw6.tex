\lhead{\textbf{Basic Algorithms, Fall 2024 \\ CSCI-UA.0310-001}}
\chead{\Large{\textbf{Homework 6}}}
\def\lc{\left\lceil}   
\def\rc{\right\rceil}
\rhead{\textbf{Instructor: Rotem Oshman \\
Name: Ishan Pranav}}
\runningheadrule
\firstpageheadrule
\cfoot{}
\subsection*{References}
Collaborated with Crystal Huang.
\subsection*{Question 1: Knapsack with repetitions}
In the lecture, you have seen the Knapsack problem: 
Given $n$ items with non-zero integer weights $w_i \in \mathbb{Z}^+$ and values $v_i \geq 0$, you want to choose a subset that maximizes the total value while the overall weight does not exceed a given bound $W$.
In the following, we want to consider a related problem where we assume that you are given $m_i$ many copies of each item type. In other words, $w_i$ and $v_i$ describe the weight and value of an item type, and you can choose that item up to $m_i$ times.

Given weights $w[1,\ldots,n]$, values $v[1,\ldots,n]$, inventory $m[1,\ldots,n]$, and an overall weight $W$, we want to maximize the overall value subject to the bound $W$. 

\begin{enumerate}
    \item Let $K[i,u]$ denote the most valuable solution using items $1,\ldots,i$ and weight bound $u$. Write a recursive formula for $K$, do not forget the base cases.

\begin{solution}
\textit{Claim. }Given weights $w_1,\dots,w_i$, values $v_1,\dots,v_1$, and inventory $m_1,\dots,m_i$,
\begin{align*}
K(i,u)=\begin{cases}
0,&i=0,\\
{\underset{0\leq j\leq\min\left(m_i,\frac{u}{w_i}\right)}{\max}}\left(K(i-1,u-jw_i)+jv_i\right)&i>0.
\end{cases}
\end{align*}
where $K(i,u)$ denotes the most valuable solution using items $1,\dots,i$ and weight bound $u\geq 0$. 

\textit{Proof. }We want to show that $K(i,u)$ denotes the most valuable solution using items $1,\dots,i$ and weight bound $u\geq 0$. We can demonstrate the claim by induction on $i$.

\textit{Basis. }Consider $i=0$. For all $u\geq 0$, the most valuable solution using the first $0$ items is $0$. Since $K(0,u)=0$ for all $u\geq 0$, the claim holds in the base case.

\textit{Hypothesis. }Consider $i=k$ where $k>0$. Assume that $K(k,u)$ denotes the most valuable solution using items $1,\dots,k$ for all weight bounds $u\geq 0$.

\textit{Inductive step. }Consider $i=k+1$. The most valuable solution maximizes the total value by considering each feasible quantity $j$ of item $k+1$.

The minimum quantity of item $k+1$ is $0$. The maximum quantity of item $k+1$ available in the inventory is given by $m_{k+1}$. However, the maximum quantity of item $k+1$ that can fit within a weight constraint of $u$ is $\frac{u}{w_{k+1}}$. Thus, the upper bound for $j$ is the smaller of $m_{k+1}$ (maximum units available in the inventory) and $\frac{u}{w_{k+1}}$ (units satisfying the weight bound), or $\min\left(m_{k+1},\frac{u}{w_{k+1}}\right)$.

For each such quantity $j$, the most valuable solution is the total weight of the $j$-many $(k+1)$-type items used, plus the most valuable solution using all the previous items $1,\dots,k$ under a smaller weight bound that excludes the total weight of the $(k+1)$-type items used. 

From the inductive hypothesis, $K(k,u-jw_{k+1})$ gives the most valuable solution for items $1,\dots,k$ under a weight bound, $u$, less the total weight of the $j$-many $(k+1)$-type items, $jw_{k+1}$.

Thus, the expression $K((k+1)-1,u-jw_{k+1})+jv_{k+1}$ gives the most valuable solution with $j$-many $(k+1)$-type items for all $u\geq 0$.

Since $k+1\geq 0$, we know that for all $u\geq 0$, the recurrence $K(k+1,u)$ gives the maximum value considering each feasible quantity of item $k+1$. This corresponds to the most valuable solution, thus completing the inductive step.

Hence, by the principle of mathematical induction, $K(i,u)$ gives the most valuable solution using items $1,\dots,i$ and weight bound $u\geq 0$.$~\square$
\end{solution}
\item Write an algorithm {\sc MaxValue}$(W,n,w,v,m)$ that computes the optimal value using Dynamic Programming. What is the running time of your algorithm?
\begin{solution}\\

\textbf{Algorithm. }{\sc MaxValue}($W,n,w,v,m$) with an overall weight bound $W\in\mathbb{Z}^+$, a number of items $n$, an $n$-element array of weights $w[1,\dots,n]$ such that $w[i]\in\mathbb{Z}^+$ for $1\leq i\leq n$, an $n$-element array of values $v[1,\dots,n]$, and an $n$-element array of quantities $m[1,\dots,n]$; returns the optimal knapsack value:

Initialize a two-dimensional array $K[0,\dots,n][0,\dots,W]$ with dimensions $(n+1)\times(W+1)$.

For $u=0$ to $W$, assign $K[0][u]\leftarrow 0$.

For $k=1$ to $n$:
\begin{itemize}
\item for $u=0$ to $W$:
\begin{itemize}
\item for $j=0$ to $\min\left(m[j],\frac{u}{w[k]}\right)$:
\begin{itemize}
\item assign $K[k]\leftarrow\max(K[k],K[k-1][u-j\cdot w[k]]+j\cdot v[k])$.
\end{itemize}
\end{itemize}
\end{itemize}
Return $K[i,W]$.\\

\textbf{Proposition. }\textit{Claim. }{\sc MaxValue}($W,n,w,v,m$) has running time $O(n\times W\times\max(m^*,W))$, where $m^*$ is the largest element in $m$.

\textit{Proof. }First, the {\sc MaxValue} algorithm initializes $W+1$ elements in $K$ with the value $0$, which is an $O(W+1)=O(W)$ process.

Then, for each of the $n$ values of $k$, for each of the $W+1$ values of $u$, the second loop takes the maximum of $\min\left(m[j],\frac{u}{w[k]}\right)+1$ values. In the worst case, the running time of the innermost loop is
\begin{align*}
O\left(\max\left(m[j],\frac{u}{w[k]}\right)+1\right)
&=O\left(\max\left(m[j],\frac{u}{w[k]}\right)\right)&\textit{asymptotically,}\\
&=O(\max(m[j],W))&\textit{in the worst case $u=W$ and $w[k]=1$.}\\
\end{align*}
Let $m^*$ be the largest element in $m$. Then, the running time of the second loop is $O(n\times W\times\max(m^*,W))$.

Therefore, the running time of {\sc MaxValue}($W,n,w,v,m$) is \[O(W)+O(n\times W\times\max(m^*,W))=O(n\times W\times\max(m^*,W)).~\square\]
\end{solution}
\end{enumerate}
\newpage
\subsection*{Question 2: Fractional knapsack}
Now consider a variant of the (original) Knapsack problem where of each item you can take a fractional part $s_i \leq w_i$ and gain value $\frac{s_i}{w_i} \cdot v_i$.
In the lecture you have seen a \emph{greedy strategy} for this problem:
\begin{itemize}
    \item Take as much as you can from the item $i$ that maximizes $\frac{v_i}{w_i}$. In other words, if $W$ is the current weight bound, take $s_i=\min(W,w_i)$ of that item.
    \item Afterwards, set $W' = W-s_i$, remove the item $i$ from consideration. Repeat until $W=0$ or there are no more items left.
\end{itemize}
In the following, we want to prove the correctness of this greedy strategy. That is, we want to prove that it maximizes the value. Assume without loss of generality that the items are sorted by decreasing $\frac{v_i}{w_i}$, i.e., that the greedy chooses the first element.
\begin{enumerate}
    \item As a warm-up, show that the greedy strategy always outputs a solution that respects the weight bound $W$.
\begin{solution}

\textbf{Algorithm I. }{\sc GreedyStrategy($W,n,w,v$)} with an overall weight bound $W\in\mathbb{Z}^+$, a number of items $n$, an $n$-element list of weights $(w_1,\dots,w_n)$ such that $w_i\in\mathbb{Z}^+$ for $1\leq i\leq n$, an $n$-element list of values $(v_1,\dots,v_n)$; return an $n$-element list of the fractional parts used, $(s_1,\dots,s_n)$, such that $s_i\in\mathbb{Q}$ and $0\leq s_i\leq w_i$. 

Assume without loss of generality that the items are sorted by decreasing value-to-weight ratio, and that the algorithm chooses the first element.

\textbf{Invariant I. }\textit{Claim. }For a given iteration $i\geq 1$, let ${s_i}_j$ denote the fractional part of item $j$ in the solution $s$, where $s_i$ denotes the solution at the beginning of iteration $i$. Similarly, let $W_i\in\mathbb{Z}^+$ denote the weight bound at the beginning of iteration $i$.

Note for each iteration $i\geq 1$ of the loop in Algorithm I, we consider item $i$. Note also iteration $i=0$ denotes the step before the loop begins. 

For every step of the loop $0\leq i\leq n$, we have \[\sum_{j=1}^i{{s_i}_j}\leq W.\] 

\textit{Proof. }We can prove this invariant by induction on $i$.

\textit{Basis. }Consider $i=0$. Note that we are considering the case that occurs before the first iteration of the loop. Since the loop body has not yet executed, \[\sum_{j=1}^i{{s_i}_j}=0\leq W,\] so the loop invariant holds.

\textit{Hypothesis. }Consider $i=k$ where $0<k<n$. Assume that \[\sum_{j=1}^k{{s_k}_j}\leq W.\]

\textit{Inductive step. }Consider $i=k+1$. Now $W_{k+1}=W_k-s_k$. Thus $W_{k+1}\leq W$.
Of course, $W_{k+1}=0$ or $W_{k+1}\neq 0$.
\begin{itemize}
\item Suppose $W_{k+1}=0$. Then the loop terminates and the loop body does not execute so \[(s_{k+1})_{k+1}=0.\] And, for all $1\leq j\leq k$, we know $(s_{k+1})_j=(s_k)_j$. From the inductive hypothesis,
\[\sum_{j=1}^{k+1}{(s_{k+1})_j}=\sum_{j=1}^k{(s_k)_j}+0\leq W.\]
\item Suppose instead $W_{k+1}\neq 0$. Then, the loop body does execute with \[(s_{k+1})_{k+1}=\min(W_{k+1},w_{k+1}).\] Now $W_{k+1}\leq w_{k+1}$ or $W_{k+1}>w_{k+1}$:
\begin{itemize}
    \item Suppose $W_{k+1}\leq w_{k+1}$. Then $(s_{k+1})_{k+1}=W_{k+1}$. So, from the inductive hypothesis, 
\[\sum_{j=1}^{k+1}{(s_{k+1})_j}=\sum_{j=1}^k{(s_k)_j}+W_{k+1}\leq W.\]
    \item Suppose instead $W_{k+1}>w_{k+1}$. Then $(s_{k+1})_{k+1}=w_{k+1}$. So, from the inductive hypothesis, \begin{align*}
    \sum_{j=1}^{k+1}{(s_{k+1})_j}&=\sum_{j=1}^k{(s_k)_j}+w_{k+1}\\&\leq W_{k+1}\\&\leq W.
    \end{align*}
\end{itemize}
\end{itemize}

In all cases of the inductive step, the loop invariant holds.

Hence, by the principle of mathematical induction, for every step of the loop $0\leq i\leq n$, we have \[\sum_{j=1}^i{{s_i}_j}\leq W.\] 

\textbf{Proposition I. }\textit{Claim. }{\sc GreedyStrategy}($W,n,w,v$) always respects the weight bound $W$.

\textit{Proof. }Let $s=\textsc{GreedyStrategy}(W,n,w,v)$. We want to show that \[\sum_{i=1}^n{s_i}\leq W.\]

Algorithm I terminates when considering item $i=n$ or when the weight bound is met.

From Invariant I, the claim holds after iteration $i=n$.

From Invariant I, the claim also holds for all iterations $0\leq i\leq n$.

Therefore, regardless of when the loop terminates, the claim holds.$~\square$
\end{solution}
\item Argue that if $\sum_{i=1}^n w_i \leq W$, the greedy strategy selects the optimal outcome.
\begin{solution}
\textbf{Proposition II. }\textit{Claim. }{\sc GreedyStrategy} selects the optimal outcome. Let $s=\textsc{GreedyStrategy}(W,n,w,v)$. Then there does not exist a solution $s'$ where $s'_k>s_k$ for some $1\leq k\leq n$ and
\[\sum_{i=1}^n\frac{s_i'}{w_i}\cdot v_i>\sum_{i=1}^n\frac{s_i}{w_i}\cdot v_i.\]

That is, there is no solution $s'$ with a greater total value constructed only by increasing at least one fractional portion in the solution.

\textit{Proof. }Assume, for the sake of contradiction, that there exists a solution $s'$ where $s'_k>s_k$ for some $1\leq k\leq n$, and $s_i'=s_i$ for all $1\leq i\leq n$ where $i\neq k$, and
\[\sum_{i=1}^n\frac{s_i'}{w_i}\cdot v_i>\sum_{i=1}^n\frac{s_i}{w_i}\cdot v_i.\]

From Algorithm I, it is given that {\sc GreedyStrategy} chooses the item with the greatest value-to-weight ratio. Algorithm I terminates when the weight bound is exhausted or when all items are considered. Thus, for all $1\leq i,j\leq n$ where $i\neq j$, if $s_i<s_j$, then $\frac{v_i}{w_i}<\frac{v_j}{w_j}$.

When {\sc GreedyStrategy} terminated and returned $s$, the weight bound was exhausted or all items were considered:
\begin{itemize}
    \item Suppose the weight bound was exhausted. Then \[\sum_{i=1}^n{s_i}=W.\] Of course,  \[\sum_{i=1}^n{s'_i}=\sum_{i=1}^n(s_i)-s_k+s_k'>W.\] This contradicts the hypothesis that $s'$ is a solution.
    \item Suppose all items were considered, but the weight bound was not exhausted. Then \[\sum_{i=1}^n{s_i}<W.\] When the weight bound is sufficient (not a binding constraint), Algorithm I chooses the full amount of each item available. From Algorithm I, since the weight bound was not exhausted, for all $1\leq i\leq n$, we have $s_i=w_i$. Now $s_k'>s_k$. Thus $s_k'>w_k$. This contradicts the hypothesis that $s'$ is a solution.
\end{itemize}

In all cases, we arrive at a contradiction.

Hence, we reject the hypothesis and conclude that there exists no such solution $s'$. Ergo the {\sc GreedyStrategy} algorithm selects the optimal outcome.$~\square$
\end{solution}
\item We call the selection $s = (s_1,\ldots,s_n)$ a strategy and let 
$\mathsf{value}(s) = \sum_{i=1}^n \frac{s_i}{w_i} \cdot v_i$ 
denote the obtained value. Let $s^*$ denote an optimal strategy (an arbitrary one if multiple exist) and $s$ the greedy strategy.
Show the following: if $s(1) \neq s^*(1)$, then there exists a valid strategy $s'$ with $\mathsf{value}(s') = \mathsf{value}(s^*)$ and $s'(1) = s(1)$. In other words, there exists an optimal strategy $s'$ that agrees with the greedy one on the first choice.
\begin{solution}
\textbf{Proposition III. }\textit{Claim. }Let $s=\textsc{GreedyStrategy}(W,n,w,v)$ and $s^*$ denote an optimal solution.  If $s_1\neq s^*_1$, then there exists a solution $s'$ where $s_1'=s_1$ and
\[\sum_{i=1}^n\frac{s_i'}{w_i}\cdot v_i=\sum_{i=1}^n\frac{s_i^*}{w_i}\cdot v_i.\]

\textit{Proof. }

\end{solution}
\item Perform an inductive argument to argue that the greedy strategy is optimal overall.
\begin{solution}
\textbf{Proposition IV. }\textit{Claim. }{\sc GreedyStrategy} is optimal overall.

Let $s=\textsc{GreedyStrategy}(W,n,w,v)$ and $s^*$ denote the overall optimal solution. Then $s$ is the overall optimal solution.

\textit{Proof. }We can demonstrate the claim by induction on $n$.

\textit{Basis. }Consider $n=1$. Then $s=(s_1)$ and $s^*=(s_1^*)$. Observe
\begin{align*}
\sum_{i=1}^1\frac{s_i}{w_i}\cdot v_i&=\frac{s_1}{w_1}\cdot v_1.\\\\
\sum_{i=1}^1\frac{s^*}{w_i}\cdot v_i
&=\frac{s_1^*}{w_1}\cdot v_1\\
&\geq\frac{s_1}{w_1}\cdot v_1&\textit{since $s^*$ is the overall optimal solution,}\\
&\geq\frac{\min(w_1,W)}{w_1}\cdot v_1&\textit{from Algorithm I, since $s_1=\min(w_1,W)$.}\\\\
s_1^*&\geq\min(w_1,W).\\\\
s_1^*&=\min(w_1,W)&\textit{if $s_1^*>w_1$ or $s_1^*>W$, then $s^*$ is not a solution,}\\
&=s_1&\textit{from Algorithm I.}
\end{align*}
Thus $s=s^*$, so $s$ is optimal overall, and the claim holds in the base case.

\textit{Hypothesis. }Consider $n=k$ where $k>1$. Then $s=(s_1,\dots,s_k)$ and $s^*=(s_1^*,\dots,s_k^*)$. Assume that $s$ is optimal overall.

\textit{Inductive step. }Consider $n=k+1$. Then $s=(s_1,\dots,s_{k+1})$ and $s^*=(s_1^*,\dots,s_{k+1}^*)$. 

Now $s_1=s_1^*$ or $s_1\neq s_1^*$:
\begin{itemize}
    \item Suppose $s_1=s_1^*$. Then $s=(s_1^*,s_2,\dots,s_{k+1})$ and $s^*=(s_1^*,\dots,s_{k+1}^*)$. Since $s_1=s_1^*$, the solution for item $1$ is known to be optimal overall. 
     \item Suppose instead $s_1\neq s_1^*$. 
     From Proposition III, we can construct a solution $s'$ where $s'_1=s_1$ and \[\sum_{i=1}^{k+1}\frac{s_i'}{w_i}\cdot v_i=\sum_{i=1}^{k+1}\frac{s_i^*}{w_i}\cdot v_i.\] Note $s'=(s_1',\dots,s_{k+1}')=(s_1,s_2',\dots,s_{k+1}')$ and $s'$ is optimal overall. Since $s_1=s_1'$, the solution for item $1$ is known to be optimal overall.
\end{itemize}
In all cases, the solution for item $1$ is known to be optimal overall.

We can consider the solution for items $2,\dots,k+1$. Note $(s_2,\dots,s_{k+1})$ is a $k$-element solution that is equivalent to the {\sc GreedyStrategy} solution for items $2,\dots,k+1$, so, by the inductive hypothesis, it is optimal overall. 

The solution for item $1$ is optimal overall, and the solution for items $2,\dots,k+1$ is optimal overall, so the solution for items $1,\dots,k+1$ (denoted $s$) is optimal overall.

Hence, by the principle of mathematical induction, $s$ is optimal overall.$~\square$
\end{solution}
\end{enumerate}
\newpage
\subsection*{Question 3: Let's Paint a Fence!}
We want to paint our new fence, which is made up $N$ boards. The
lengths of the $N$ boards are given in an array
$L[1,\ldots,N]$. We have hired $K$ painters, and
each painter takes $1$ hour to paint a $1$ unit of the board. For example,
if one of the painters paints boards $3$, $4$ and $5$, then they
complete at time $t_i=L[3]+L[4]+L[5]$. Our goal is to assign each
painter to some subset of boards, to minimize the time when the fence
has been completely painted. Since the $K$ painters can work in
parallel, this corresponds to minimizing $\max(t_1,\ldots,t_K)$, where
$t_i$ is the time taken by painter $i$ to complete their job. The
painting task must be accomplished under the following constraints:
\begin{itemize}
    \item Each board must be completely painted by exactly one painter;
      i.e., no board can be painted partially by one painter and partially
      by another.
    \item Each painter paints a contiguous collection of boards.  For example, a
      configuration where painter $1$ paints boards $1$ and $3$ but not
      $2$ is not a valid solution.
\end{itemize}
You are given as input the following: the number of painters $K$, and
an array $L$ with the board lengths.  In the following problems,
denote by $T(i,j)$ the minimum time to paint the first $i$ boards using
$j$ painters. Using Dynamic Programming, we want to come up with a procedure 
to minimize the painting time.
\begin{enumerate}
\item Write a recurrence for $T(i,j)$ in terms of $T(*,j-1)$, i.e., of entries $T(k,j-1)$ for arbitrary $k$.
\begin{solution}
\textit{Claim. }Let $L[i]$ denote the length of board $i$ for $1\leq i\leq N$. Then for all $0\leq i\leq N$,
\begin{align*}
T(i,j)=\begin{dcases}
0,&i=0,\\
+\infty,&i>0\text{ and }j=0,\\
\sum_{k=1}^iL(k),&i>0\text{ and }j=1,\\
\min_{0\leq k\leq i}(\max(T(i-k,j-1),T(k,1))),&i>0\text{ and }j>1.
\end{dcases}
\end{align*}
denotes the minimum time required to paint the first $i$ boards using $1\leq j\leq K$ painters. 

\textit{Proof. }We can demonstrate the claim by induction on $j$.

\textit{Basis. }
Consider $j=0$. Then $i=0$ or $i>0$:
\begin{itemize}
    \item Suppose $i=0$. There are no boards, so the minimum time required is $0$. Of course, $T(0,0)=0$, so the claim holds in this case.
    \item Suppose instead $i>0$. There are boards but no painters, so the minimum time required to paint the first $i$ boards is unbounded ($+\infty$). Of course, $T(i,0)=+\infty$ for all such $i$, so the claim holds in this case.
\end{itemize}
Consider $j=1$. Then $i=0$ or $i>0$:
\begin{itemize}
    \item Suppose $i=0$. Then there are no boards, so the minimum time required is $0$. Of course, $T(0,1)=0$, so the claim holds in this case.
    \item Suppose instead $i>0$. Then there is only one painter, so the time required to paint the first $i$ boards is the sum of the times required to paint each board. There is no parallelization in this case. Note $T(i,1)$ gives this result for all $0<i\leq N$, so the claim holds.
\end{itemize}
The claim holds in all base cases.

\textit{Hypothesis. }Consider $j=\ell$ where $\ell>1$. Assume for all $0\leq i\leq N$ that $T(i,\ell)$ gives the minimum time required to paint the first $i$ boards using $\ell$ painters.

\textit{Inductive step. }Consider $j=\ell+1$. For all $0\leq i\leq N$, the optimal solution for $i$-many boards minimizes the required time by considering each feasible contiguous number of boards $k$ that painter $(\ell+1)$ may paint.

The minimum value of $k$ is $0$, since if $j>i$, a single painter may be assigned to paint no boards. The maximum value of $k$ is $i$. Since $\ell>1$, this case occurs when $i=1$ so a single painter is assigned to paint all $i$ boards.

For each such quantity of boards $k$, the minimum time required to paint the first $k$ boards is the larger of the minimum time required for the single painter $(\ell+1)$ to paint the last $k$-many boards, and of the minimum time required for the remaining $\ell$ painters to paint the first $(i-k)$ boards.

From the inductive hypothesis, $T(i-k,(\ell+1)-1)$ gives the minimum time required for $\ell$-many painters to paint $(i-k)$-many boards and $T(k,1)$ gives the minimum time required for $1$ painter to paint $k$-many boards.

Thus, the expression $\max(T(i-k,(\ell+1)-1),T(k,1))$ gives the optimal solution where painter $(\ell+1)$ paints $k$-many boards.

Since $\ell+1\geq 1$, we know that for all $0\leq i\leq N$, the recurrence $T(i,\ell+1)$ gives the minimum value considering each feasible number of boards $k$. This corresponds to the optimal solution, thus completing the inductive step.

Hence, by the principle of mathematical induction, $T(i,j)$ gives the minimum time required to paint the first $0\leq i\leq N$ boards using $1\leq j\leq K$ painters.$~\square$
\end{solution}	
    \item Write an algorithm {\sc MinTime} that computes the minimal completion time. To this end, complete the following skeleton. Make sure your algorithm uses only the minimal amount of extra storage.
	
    \begin{code}
        1 {\sc MinTime}($N,K$)\\
        2 \> $T \gets $ new array of length $\ldots\ldots\ldots$ \\
        3 \> $T[\ldots\ldots] \gets \ldots\ldots$ \\
        4 \> \For $\ldots\ldots=1$ \To $\ldots\ldots$ \Do \\
        5 \> \> $T[\ldots\ldots] \gets \ldots\ldots\ldots$ \\
        6 \> \For $\ldots\ldots=1$ \To $\ldots\ldots\ldots$ \Do \\
        7 \> \> \For $\ldots\ldots=\ldots\ldots$ \DownTo $1$ \Do \\
        8 \> \> \>  $T[\ldots\ldots] \gets \text{\sc Calc}(T,L,N,K,i,j)$ \\
        9 \> \Return $T[\ldots\ldots]$
    \end{code}

\begin{solution}

\textbf{Algorithm 1. }{\sc MinTime}($L,N,K$) with an $N$-element array $L[1,\dots,N]$, where $L[i]$ denotes the length of board $1\leq i\leq N$, and a number of painters $K\in\mathbb{N}$; return the minimum time required to paint $N$ boards with $K$ painters:

Initialize a two-dimensional array $T[0,\dots,N][0,\dots,K]$ with dimensions $(N+1)\times(K+1)$.

Assign $T[0][0]\leftarrow 0$.

For $j=1$ to $K$, assign $T[0][j]\leftarrow 0$.

For $i=1$ to $N$:
\begin{itemize}
    \item assign $T[i][0]\leftarrow+\infty$;
    \item assign \[T[i][1]\leftarrow\sum_{k=1}^i{L[k]}.\]
\end{itemize}

For $i=1$ to $N$:
\begin{itemize}
    \item for $j=2$ to $K$, assign $T[i][j]\leftarrow\textsc{Calc}(T,L,N,K,i,j)$.
\end{itemize}
Return $T[N][K]$.

\textbf{Proposition 1. }\textit{Claim. }{\sc MinTime} uses the minimal amount of extra storage.

\textit{Proof. }
\end{solution}
\item Next, write an algorithm $\text{\sc Calc}(T,L,N,K,i,j)$ that calculates $T[i,j]$
according to your recursive formula from part~(1). Your algorithm should run in time $O(N)$.
\begin{solution}

\textbf{Algorithm 2. }{\sc Calc}($T,L,N,K,i,j$) given a two-dimensional array\\ $T[0,\dots,N][0,\dots,K]$ with dimensions  $(N+1)\times (K+1)$, an $N$-element array $L[1,\dots,N]$, where $L[\ell]$ denotes the length of board $1\leq\ell\leq N$, the subproblem number of boards\\ $1\leq i\leq N$, and the subproblem number of painters $1\leq j\leq K$:

Assign $T[i,j]\leftarrow+\infty$.

For $k=0$ to $i$, assign $T[i,j]\leftarrow\min(T[i,j],\max(T[i-k,j-1],T[k,1])$.

\textbf{Proposition 2. }\textit{Claim. }{\sc Calc}($T,L,N,K,i,j$) has running time $O(N)$.

\textit{Proof. }First, the algorithm assigns $+\infty$ to $T[i,j]$, which is an $O(1)$ process.

Then, for each of the $i+1$ indices from $k=0$ to $k=i$, the algorithm assigns a computed value to $T[i,j]$. The inner loop body computation is an $O(1)$ process. The maximum value for $i$ is $N$. Thus, the worst-case running time of the loop is $O(1\cdot N)=O(N)$.

Therefore, the running time of {\sc Calc}($T,L,N,K,i,j$) is $O(1)+O(N)=O(N).~\square$
\end{solution}
\end{enumerate}